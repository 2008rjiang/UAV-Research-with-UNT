\documentclass[letterpaper,11pt]{article}

\usepackage{longtable}
\renewcommand{\familydefault}{\sfdefault}

\title{Air Corridor Multi-agent Simulation}
%\author{P\'eter Moln\'ar}

\pagestyle{headings}

\begin{document}
\maketitle

\begin{abstract}
    The rapid growth of unmanned air vehicles (UAVs) in applications such as delivery services, surveillance, and disaster response necessitate efficient and safe navigation strategies in three-dimensional space. Inspired by the concept of self-organized trail formation in pedestrian dynamics, this project aims to develop a computer simulation for self-organized air corridor generation in arbitrary dimensions. Building on the social force model\cite{PhysRevE.51.4282}, the simulation will model UAVs as agents influenced by social forces to form trails that optimize efficiency and safety. The goal is to create adaptive air corridors that account for environmental factors, geographical constraints, and UAV-specific behaviors.
\end{abstract}


\section{Objective of Air Corridors in Minimizing Airspace Usage and Environmental Impact}

The primary objective of air corridors is to optimize the use of airspace by concentrating air traffic along well-defined routes, thereby reducing the environmental impact of aviation without compromising safety or significantly extending flight paths. By channeling aircraft into structured pathways, air corridors help minimize the dispersion of flights across large areas, which reduces the overall volume of airspace that must be actively managed. This structured approach not only enhances operational efficiency but also mitigates the environmental footprint of air traffic by enabling shorter, more direct routes and minimizing unnecessary deviations.

Air corridors contribute to environmental sustainability by reducing fuel consumption and emissions. Aircraft flying within these corridors can follow optimized trajectories with fewer delays, as the structured nature of the corridors minimizes conflicts and bottlenecks. For example, free route airspace and performance-based navigation technologies allow aircraft to plan efficient, stable trajectories that reduce fuel burn and emissions. Additionally, by concentrating traffic along defined routes, noise pollution is localized to specific areas, reducing its impact on broader communities. This approach aligns with modern air traffic management strategies that aim to balance operational efficiency with environmental considerations.

Moreover, air corridors ensure safety by providing predictable and controlled environments for air traffic. They facilitate collision avoidance through predefined separation standards and allow for better integration of unmanned aerial vehicles (UAVs) into shared airspace. The structured nature of air corridors reduces the complexity of managing traffic flows, requiring fewer tactical interventions from air traffic controllers while maintaining high levels of safety. %Overall, air corridors represent a crucial element in modernizing airspace management to meet the dual challenges of increasing air traffic demand and reducing aviation's environmental impact.

\section{Background}

\subsection{The Social Force Model}
The \textbf{Social Force Model (SFM)} %, introduced by Dirk Helbing and Péter Molnár in the mid-1990s,
is a mathematical framework designed to describe and simulate the movement of pedestrians and their collective behaviors. The model conceptualizes pedestrian motion as being influenced by "social forces," which are not physical forces in the classical sense but rather represent internal motivations that drive individuals to perform specific actions, such as moving toward a destination or avoiding collisions. These forces are encoded as terms in a set of equations that govern the dynamics of pedestrian behavior. The model's key innovation lies in its ability to capture self-organized phenomena, such as lane formation in crowds or the emergence of trails in open spaces, through simple interactions between agents and their environment.

At its core, the SFM incorporates three primary types of forces: (1) a \textit{driving force} that propels individuals toward their desired velocity and destination, (2) \textit{repulsive forces} that prevent collisions with other pedestrians or obstacles by maintaining a safe distance, and (3) \textit{attractive forces} that account for group cohesion or the influence of environmental features like landmarks. These forces are combined into nonlinearly coupled Langevin equations, enabling realistic simulations of pedestrian dynamics. The model has been widely applied in scenarios such as urban planning, evacuation modeling, and crowd management. Its versatility has also inspired numerous extensions, including adaptations for dynamic route choices, group behaviors, and even non-pedestrian systems like vehicular traffic or UAV navigation. By capturing the interplay between individual motivations and collective patterns, the SFM has become a foundational tool for studying self-organization in complex systems.

\subsection{Trail Formation in the Social Force Model}

Trail formation in the Social Force Model (SFM) is implemented by extending the basic framework to account for feedback mechanisms between agents and their environment \cite{Helbing1997,helbing1998computersimulationspedestriandynamics}. In this extended model, often referred to as an "active walker" model, pedestrians not only react to their surroundings but also modify it dynamically as they move. Specifically, as agents traverse a space, they leave behind a "trail potential" that influences the movement of subsequent agents. This trail potential represents the attractiveness of frequently used paths, encouraging other agents to follow these routes. Over time, this feedback mechanism leads to the emergence of self-organized trail networks that reflect the collective behavior of the agents.

The topological structure of the resulting trail system depends on several factors, such as the cost of creating new trails and the durability or persistence of existing ones. For example, if creating new paths is costly or difficult, agents are more likely to consolidate their movements along existing trails, leading to well-defined pathways. Conversely, if trails decay quickly or are easily modified, the system may produce more dynamic and less stable trail patterns. Simulations using this approach have demonstrated its ability to reproduce observed features of real-world trail systems, such as those found in parks or urban green spaces. These results highlight the utility of the model for applications like urban planning, where optimizing pedestrian pathways can improve accessibility and efficiency.



\section{Adapting the Social Force Model for UAV Air Corridors}

To simulate the formation of air corridors for unmanned aerial vehicles (UAVs), the Social Force Model (SFM) can be extended to three-dimensional space and adapted to incorporate factors unique to aerial navigation. This adaptation involves modeling UAVs as agents influenced by forces that account for their goals, interactions, and environmental constraints, while capturing the emergent behavior of self-organized trail systems.

Locations for take-off and landing are fixed positions
in three-dimensional space that exert strong attractive forces on UAVs, guiding them toward their destinations.
These forces can dynamically adjust based on operational factors such as time schedules or UAV priorities,
ensuring that the system accommodates varying levels of urgency or mission-critical tasks.
This allows the model to simulate realistic scenarios where UAVs must coordinate their movements while adhering to predefined objectives.

The UAVs themselves are modeled as agents with specific parameters, including speed, size, and maneuverability. Driving forces propel these agents toward their destinations while avoiding collisions with other UAVs or obstacles. To ensure safe separation, repulsive forces are incorporated into the model, which depend on the relative positions and velocities of neighboring UAVs. These forces help prevent mid-air collisions and maintain orderly traffic flow within the airspace.

Environmental factors such as wind and weather conditions are also integrated into the model. Wind is represented as an external force field that acts on UAVs, influencing their trajectories based on its strength and direction, which can vary spatially and temporally. Additional external forces or constraints can simulate weather phenomena like turbulence or zones of restricted visibility, further enhancing the realism of the simulation.

Geographical constraints play a crucial role in shaping UAV behavior within the model. No-fly zones are represented as regions in three-dimensional space that exert strong repulsive forces to prevent UAVs from entering restricted areas. Terrain features such as mountains or buildings are similarly modeled with repulsive forces proportional to proximity, ensuring that UAVs navigate safely around obstacles. Additionally, predefined flight paths or high-priority corridors can exert attractive forces to encourage alignment with existing air traffic patterns or optimize overall system efficiency.


\section{Simulation Program}

The simulation program for self-organized air corridor formation will be developed in Julia using the \texttt{Agents.jl} package \cite{Agents.jl}. \texttt{Agents.jl} provides a flexible framework for agent-based modeling (ABM), allowing the implementation of autonomous agents (UAVs) and their interactions within a continuous three-dimensional space. The Social Force Model (SFM) will be integrated into this framework to govern the behavior of UAVs, enabling the emergence of optimal air corridors. Below is a detailed plan for implementing the simulation program.

The Social Force Model will serve as the core mechanism driving UAV behavior. Each UAV will be influenced by a combination of forces: a driving force that propels it toward its destination, repulsive forces that ensure collision avoidance with other UAVs and obstacles, and environmental forces such as wind or no-fly zones. These forces will be computed dynamically during each time step and used to update the UAV's position and velocity. The SFM equations will be implemented within the \texttt{step\_agent!} function provided by \texttt{Agents.jl}, which defines how individual agents behave during each simulation step. This modular approach ensures that the SFM can be easily customized or extended as needed.

The UAVs will be modeled as agents in the \texttt{Agents.jl} framework, with attributes such as position, velocity, destination, size, and maneuverability. These attributes will allow the simulation to capture realistic UAV dynamics and interactions. The \texttt{ContinuousSpace} type in \texttt{Agents.jl} will be used to represent the three-dimensional airspace in which UAVs operate. This space will enable smooth movement and precise positioning of agents within a continuous 3D grid. The grid boundaries will define the operational limits of the airspace, while obstacles such as no-fly zones or terrain features can be represented as regions within this space that exert strong repulsive forces on nearby UAVs.

Environmental factors such as wind fields and weather conditions will be incorporated into the simulation as external forces acting on UAVs. Wind can be modeled as a spatially varying vector field that influences UAV trajectories based on their positions. Similarly, no-fly zones will be implemented as fixed regions in the 3D space that apply repulsive forces proportional to proximity, ensuring that UAVs avoid restricted areas. These environmental effects will be updated dynamically during each simulation step to reflect changing conditions.

The simulation workflow will involve initializing a population of UAV agents with randomly assigned start and landing points within the 3D grid. During each time step, the \texttt{step\_agent!} function will compute the net force acting on each UAV based on its current state and environment, updating its position and velocity accordingly. The \texttt{step\_model!} function will handle global updates to environmental factors, such as changes in wind patterns or weather conditions. The simulation will run until all UAVs have reached their destinations or a predefined time limit is reached.

% By leveraging \texttt{Agents.jl}, this approach ensures modularity, scalability, and flexibility in implementing the Social Force Model for UAV air corridor formation. The use of Julia's high-performance capabilities further allows efficient execution of large-scale simulations on multi-core CPUs or distributed systems.


\section{Creating Air Corridors from Self-Organized Trails Using Three-Dimensional Hough Transform}

The three-dimensional Hough transform (3DHT) is a powerful technique for detecting lines or trajectories in 3D space and can be applied to extract air corridors from self-organized trails generated by the Social Force Model (SFM). The SFM produces trails as emergent phenomena based on the interactions of unmanned aerial vehicles (UAVs) with their environment. These trails, represented as point clouds in 3D space, can be processed using the 3DHT to identify dominant linear patterns that correspond to potential air corridors. This approach leverages the ability of the Hough transform to detect parametric shapes, such as straight lines, by identifying clusters of points that align along specific directions in a discretized parameter space.

In the 3DHT, a line is typically represented in vector form as \(\vec{a} + t\vec{b}\), where \(\vec{a}\) is a point on the line, \(\vec{b}\) is the direction vector of the line (normalized to unit length), and \(t\) is a scalar parameter. The parameter space for detecting lines in 3D consists of three dimensions: two angles (\(\phi\) for azimuth and \(\theta\) for elevation) that define the orientation of \(\vec{b}\), and a distance parameter \(r\) that defines the perpendicular distance from the origin to the line. The input to the 3DHT is a set of points from the trail system, which are accumulated into bins in this parameter space. Each bin represents a potential line, and its value reflects the number of points that align with that line. By identifying peaks in the accumulator array, which correspond to local maxima, dominant lines can be extracted.

To apply this method for air corridor formation, the UAV trajectories generated by the SFM are first sampled as discrete points in 3D space. These points are then fed into the 3DHT algorithm, which iteratively detects prominent lines by voting in the parameter space. Post-processing techniques, such as thresholding and smoothing, are applied to refine these detected lines and eliminate noise or spurious detections caused by scattered data points. The result is a set of well-defined linear paths that represent candidate air corridors. These corridors can then be evaluated based on criteria such as traffic density, safety margins, and environmental constraints.

The 3DHT-based approach ensures that air corridors are derived directly from UAV behavior and environmental interactions modeled by the SFM. This integration allows for adaptive corridor formation that reflects real-world conditions while optimizing airspace usage and minimizing environmental impact. By combining self-organized trail dynamics with robust line detection techniques, this method provides a scalable solution for designing efficient and safe UAV navigation systems.

\section{Summary of Concepts}
\begin{longtable}{|p{0.20\textwidth}|p{0.40\textwidth}|p{0.40\textwidth}|}
    \hline
    \textbf{Concept} & \textbf{Description} & \textbf{Implementation} \\ \hline
    \endfirsthead
    \hline
    \textbf{Concept} & \textbf{Description} & \textbf{Implementation} \\ \hline
    \endhead
    \hline
    \endfoot
    
    Social Force Model & A mathematical framework that models the movement of agents (e.g., pedestrians or UAVs) as being influenced by internal motivations and external forces. & Implemented using differential equations to calculate driving, repulsive, and attractive forces acting on each agent at every time step. \\ \hline
    
    UAV Agents & Autonomous aerial vehicles modeled as agents with attributes such as position, velocity, and destination. & Each UAV is represented as an agent in the \texttt{Agents.jl} framework with state variables updated dynamically based on the Social Force Model. \\ \hline
    
    3D Continuous Space & The environment in which UAVs operate, represented as a continuous three-dimensional grid. & Implemented using the \texttt{ContinuousSpace} type in \texttt{Agents.jl}, allowing smooth movement and precise positioning of agents in 3D. \\ \hline
    
    Environmental Factors & External influences such as wind fields, weather conditions, or no-fly zones that affect UAV trajectories. & Modeled as external force fields or regions in the 3D space that apply repulsive or directional forces to UAVs. \\ \hline
    
    Air Corridors & Structured pathways in 3D space that optimize air traffic flow while minimizing environmental impact and ensuring safety. & Derived from self-organized trails using the three-dimensional Hough transform to detect linear patterns in UAV trajectories. \\ \hline
    
    Trail Formation & Emergent paths created by the repeated traversal of agents through specific regions of space. & Implemented by dynamically modifying the environment based on agent movement, creating attractive potentials along frequently used paths. \\ \hline
    
    Collision Avoidance & Mechanism to ensure safe separation between agents and prevent mid-air collisions. & Achieved using repulsive forces in the Social Force Model that depend on relative positions and velocities of neighboring UAVs. \\ \hline
    
    Simulation Workflow & The process of initializing agents, updating their states, and running the simulation over time steps. & Implemented in Julia using the \texttt{Agents.jl} framework with functions like \texttt{step\_agent!} for individual updates and \texttt{step\_model!} for global updates. \\ \hline
    
    Visualization & Tools to monitor and analyze the behavior of agents and emergent patterns during the simulation. & Real-time visualization implemented using Julia plotting libraries like \texttt{CairoMakie} for 3D rendering of UAV trajectories and air corridors. \\ \hline
    
    Scalability & The ability of the simulation to handle large numbers of agents or complex environments efficiently. & Achieved using Julia's multi-threading, distributed computing capabilities, and GPU acceleration where necessary. \\ \hline
    
\end{longtable}



\section{Implementation Plan}
\begin{longtable}{|p{0.20\textwidth}|p{0.80\textwidth}|}
    \hline
    \textbf{Workstream} & \textbf{Description}  \\ \hline
    \endfirsthead
    \hline
    \textbf{Workstream} & \textbf{Description} \\ \hline
    \endhead
    \hline
    \endfoot

Simulation Framework & Using programming language \texttt{Julia} with \texttt{Agents.jl} framework. Using docker-ized environment to deploy in different compute environments. \\ \hline


\end{longtable}

\nocite{*} %% remove in final version
\bibliographystyle{plain}
\bibliography{references}
\end{document}

